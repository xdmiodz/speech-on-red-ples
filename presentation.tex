\documentclass[10pt]{beamer}
\setbeamertemplate{navigation symbols}{}
\usetheme{Warsaw}
\usepackage[warn]{mathtext}
\usepackage{amsmath,amssymb}
\usepackage{ragged2e}
\justifying
\beamersetuncovermixins{\opaqueness<1>{25}}{\opaqueness<2->{15}}
\usepackage[english, russian]{babel}
\usepackage{pscyr}
\renewcommand{\rmdefault}{ftm}
\usepackage[utf8]{inputenc}
\usepackage{graphicx}
\usepackage{multirow}
\usepackage{array,hhline,longtable}
\addto\captions\russian{%
\renewcommand{\figurename}{}
\renewcommand{\tablename}{}%
}
\renewcommand{\raggedright}{\leftskip=0pt \rightskip=0pt plus 0cm}
\title[Параметрическая генерация ОНЧ волн в плазме]{Параметрическая генерация низкочастотных волн электронами плазмы, ускоренными в условиях электронного циклотронного резонанса}

\begin{document}

\maketitle

\section{Введение}
\begin{frame}\frametitle{Роль НЧ волн в околоземной Земли}
  \small
  \begin{itemize}
  \item Низкочастотные волны $\left(100\,Гц\div10\,кГц\right)$, излучаемые искусственными и природными источниками, применяются как для диагностики и мониторнинга плазменного окружения Земли~\cite{AWD,IMAGE}, так и активного воздействия на процессы, происходящие в нем~\cite{Demekhov, InanBellBortnik}.
  \item Эффективность возбуждения свистовых волн с использованием компактных спутниковых антенных систем невелика вследствие больших длин волн излучения.
  \item  Один из путей повышения эффективности генерации низкочастотных волн состоит в создании протяженных излучающих токовых систем непосредственно  в околоземной плазме~\cite{Holzworth, HAARP}.
  \end{itemize}
  \begin{thebibliography}{9}
    \tiny
  \bibitem{AWD}\textit{Rodger C.J.,  Lichtenberger J., McDowell G., Thomson N.R.} // Radio Science. 2009. Volume 44, Issue 2. %found
  \bibitem{IMAGE}\textit{Burch J.L.} // Space Science Reviews. 2000.  Volume 91, pp. 1--14. %found
  \bibitem{Demekhov}\textit{Demekhov A.G., Trakhtengerts V.Y., Mogilevsky M.M., and Zelenyi L.M.} // Adv.~Sp.~Res. 2003. Volume 32, p. 355. %found
  \bibitem{InanBellBortnik}\textit{Inan U.S., Bell T.F., Bortnik J. et al.} // J.~Geophys.~Res. 2003. Volume 108, p. 1186.%found
  \bibitem{Holzworth}\textit{Holzworth R.H., Koons H.C.} // J.~Geophys.~Res. 1981. Volume 86, pp. 853--857 
  \bibitem{HAARP}\textit{Platino M., Inan U.S., Bell T.F., et al.} // Ann.~Geophys. 2004. Volume 22, p. 2643. %found
  \end{thebibliography}
\end{frame}

\begin{frame}\frametitle{Ускорение электронов в ближней зоне антенных систем космического базирования}
  \begin{columns}
    \begin{column}{6cm}
      \small
      В данной работе  предлагается использовать параметрический метод генерации волн свистового диапазона с использованием традиционных антенных систем, размещаемых на КА. В качестве основного механизма генерации предлагается использовать эффект резонансного ускорения электронов под воздействием амплитудно-модулированной ВЧ накачки в условиях электронного-циклотронного резонанса~\cite{Galperin}.
      \vspace{1cm}
\begin{thebibliography}{9}
    \tiny
  \bibitem{Galperin}\textit{Shuiskaya F.K., Galperin Yu.I., Serov A.A., et al.} // Planet.~Sp.~Sci. 1990. Volume 38, p. 173. %found

  \end{thebibliography}
    \end{column}
    \begin{column}{5cm}

      \begin{figure}
        \centering
        \small
        \def\svgwidth{130pt} % sets the image width, this is optional
        \input{intercosmos.tex}
      \end{figure}
    \end{column}
  \end{columns}
  
\end{frame}

\begin{frame}\frametitle{Параметрическая антенна}
  \small
  \begin{figure}
    \centering
    \small
    \def\svgwidth{250pt} % sets the image width, this is optional
    \input{bodyless.tex}
  \end{figure}
  \begin{itemize}
    \small
  \item Из-за диамагнитного эффекта в возмущенной области плазмы формируется крупномасштабная система замкнутых поперечных токов – реализуются условия для возбуждения НЧ волн электромагнитного типа (например, свистовых или альфвеновских).
  \item  Имеется возможность перестройки частоты НЧ излучения в широком диапазоне.  
  \end{itemize}
\end{frame}

%\begin{frame}\frametitle{Общая характеристика волн свистового диапазона частот}
%  \small
%  \begin{itemize}
%  \item Волны свистового дипазона представляют собой низкочастотные электромагнитные волны с правой эллиптической поляризацией. Границы свистового диапазона определяются следующими неравенствами:
%    \[\Omega\ll\omega_{LH}<\omega<\omega_{H}\ll\omega_p\]
%$\Omega_{H}$ --- циклотронная частота ионов, $\omega_{LH}=\left(\Omega_H\omega_H\right)^{1/2}$  --- частота нижнегибридного резонанса. 
%\item Дисперсионное соотношение для волн свистового диапазона при произвольном направлении волнового вектора имеет следующий вид: \[n^2 = \frac{k^2c^2}{\omega^2}=1+\frac{\omega^2_p}{\omega\left(\omega_H\cos\Theta-\omega\right)}\]
%$\Theta$ --- угол между направлением внешенего магнитного поля $\vec{B}$ и волновым вектором свистов $\vec{k}$.
%  \end{itemize}
%\end{frame}

\section{Лабораторные эксперименты}
%\begin{frame}\frametitle{Плазменный стенд <<Крот>>}
%  \small
%  \begin{tabular}{|l|l|}
%    \hline
%    \multirow{4}{*}{Индукционный ВЧ источник плазмы} & f = 4 МГц\\
%     & 4 генератора\\
%     & P = 1 МВт каждый\\
%     & $t_p=1.5$\,мс\\\hline
%    \multirow{3}{*}{Источник магнитного поля} & Емкостной накопитель\\
%    & $E=1$\,МДж\\
%    & $B_0=1\div1000$\,Гс\\\hline
%    Предельный вакуум & $3\cdot{}10^{-6}$\,Торр\\
%    \hline
%    Рабочий газ & $Ar$, $He$\\\hline
%    Давление рабочего газа & $p=1\cdot{}10^{-4}\div{}5\cdot{}10^{-3}$\,Торр\\\hline
%    Размеры замагниченной плазмы & длина 5 м, диаметр 1.5 м\\\hline
%    Концентрация плазмы & $n_{e}=10^{6}\div{}10^{13}$\,см$^{-3}$\\\hline
%    Температура электронов & $T_e=0.1\div{}10$\, эВ\\\hline
%  \end{tabular}
%\end{frame}

\begin{frame}\frametitle{Параметры модельного лабораторного эксперимента}
   \small
   Масштабный множитель $\gamma$ вводится как отношение характерных пространственных масштабов исследуемого явления в космосе и в лаборатории~\cite{Alven}.
   \begin{table}
   \begin{tabular}{|m{3.7cm}|m{1.8cm}|m{3.5cm}|}
     \hline
     \textbf{Параметр} & \textbf{Ионосфера} & \textbf{Лабораторная плазма}\\\hline
     Концентрация электронов $n_{e}$,\,см$^{-3}$ & $10^{3}\div{}10^{6}$ & $10^{7}\div{}10^{10}$ ($10^{6}\div{}10^{11}$)\\\hline
     Температура электронов $T_{e}$,\,эВ & $0.2\div5$ & $0.2\div5$ ($0.3\div3$)\\\hline
     Индукция статического магнитного поля $B$,\,Гс & $0.2\div0.5$ & $20\div50$ ($10\div100$)\\\hline
     Размер антенны $L$,\,см & $1000$ & $10$ ($7$)\\\hline
     Частота $f$,\,МГц &$0.1\div10$& $10\div{}1000$ ($65\div 80$)\\\hline
     Мощность передатчика $P$,\,Вт &$100\div1000$& $100\div1000$ ($300$)\\\hline
   \end{tabular}
\caption{Параметры активных ионосферных и модельных лабораторных экспериментов при значении масштабного множителя $\gamma = 100$; в скобках приведены фактические параметры эксперимента на стенде <<Крот>>.}
\begin{thebibliography}{9}
    \tiny
\bibitem{Alven} Г.~Альвен, К.-Г.~Фельтхаммар, Космическая электродинамика
\end{thebibliography}
\end{table}
  
\end{frame}

\begin{frame}\frametitle{Описание экспериментов}
  \begin{columns}
    \begin{column}{5cm}
      \begin{itemize}
        \item Измерение концентрации плазмы --- зонд с СВЧ-резонатором~\cite{Stenzel,Yanin}.
        \item Измерение электронной температуры --- двойной зонд~\cite{UHF_probe}. 
      \end{itemize}
  \vspace{1cm}
      \begin{thebibliography}{9}
        \tiny
      \bibitem{Stenzel}\textit{Stenzel R.L.}  // Rev.~.~Instrum. 1976. Volume 47, p. 603.
        
      \bibitem{Yanin}\textit{Янин Д.В., Костров А.В., Смирнов А.И., Стриковский А.В.} // ЖТФ. 2008. т. 78, стр. 133.
        
      \bibitem{UHF_probe}\textit{Кондратьев И.Г., Костров А.В., Смирнов А.И., Стриковский А.В., Шашурин А.В.} Резонансный СВЧ-зонд  на отрезке двухпроводной линии: препринт № 585 // ИПФ РАН. Н. Новгород. 2001. 23 c.
      \end{thebibliography}
      \vspace{4cm}
    \end{column}
    \begin{column}{5cm}
      \begin{overprint}
        \includegraphics[scale=0.23]{scheme_setup}
      \end{overprint}
     \vspace{3cm}
    \end{column}
  \end{columns}
\end{frame}

\section{Экспериментальные результаты}
\begin{frame}\frametitle{Диамагнитный сигнал в плазме}
  \begin{columns}
    \begin{column}{6cm}      
      \begin{figure}
        \centering
        \small
        \def\svgwidth{160pt} % sets the image width, this is optional
        \input{pulse_demo.tex}
      \end{figure}
    \end{column}
    \begin{column}{6cm}
      \begin{figure}
        \centering
        \small
        \def\svgwidth{160pt} % sets the image width, this is optional
        \input{fig1.tex}
      \end{figure}
      \vspace{1.5cm}
    \end{column}
  \end{columns}
\end{frame}

\begin{frame}\frametitle{Поперечное распределение амлитуды НЧ волн}
  \begin{columns}
    \begin{column}{6cm}      
      \small
      \begin{itemize}
        \item $(а)$ Радиальное распределение амплитуды НЧ волн свистового диапазона, возбуждаемых амплитудно-модулированным ВЧ сигналом в условиях ЭЦР ($f=66.5$\,МГц $\simeq f_{ce}$, $P\simeq 300$\,Вт) при различных частотах модуляции $f_{m}$, и регистрируемых на расстоянии $z=48$\,см от антенны. 
          \item $(б)$ Радиальное распределение амплитуды пробных НЧ волн, возбуждаемых при непосредственной подаче НЧ сигнала на рамочную антенну; измерения выполнены на тех же частотах $F=f_m$ в том же сечении $z$.
        \end{itemize}
      \vspace{3cm}
    \end{column}
    \begin{column}{7cm}
      \begin{figure}
        \centering
        \footnotesize
        \def\svgwidth{130pt} % sets the image width, this is optional
        \input{param_vs_dir.tex}
      \end{figure}
      \vspace{3cm}
    \end{column}
  \end{columns}
\end{frame}

\begin{frame}\frametitle{Свойства генерируемых НЧ волн}
  \begin{columns}
    \begin{column}{5cm}
      \begin{figure}
        \centering
        \tiny
        \def\svgwidth{105pt} % sets the image width, this is optional
        \input{phase_composite.tex}
      \end{figure}
     \vspace{3cm}
      \small
    \end{column}
 \begin{column}{6cm}      
      \small
      \begin{figure}
        \centering
        \tiny
        \def\svgwidth{110pt} % sets the image width, this is optional
        \input{h_separation.tex}
      \end{figure}
          \vspace{3cm}
     \end{column}
  \end{columns}

\end{frame}

\section{Результаты численного моделирования}
\begin{frame}
\end{fram}
\end{document}
